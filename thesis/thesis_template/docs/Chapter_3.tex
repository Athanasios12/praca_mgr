\documentclass[document.tex]{subfiles}
\begin{document}
\chapter{Wyniki badań doświadczalnych \\ implementacji algorytmu Viterbiego}
\indent Do sprawdzenia poprawności algorytmu Viterbiego do detekcji linii, wykorzystano
zdjęcia linii z różnym poziomem zaszumienia. W celu porównania szybkości
działania poszczególnych implementacji algorytmu zdefiniowano zestaw zdjęć testowych
o różnym rozmiarze. Do przedstawienia wyników zestawień parametrów algorytmu Viterbiego
dla jego różnych wersji, napisana została funkcja automatycznie generująca plik .csv,
zawierający tabelę z parametrami wejściowymi i zestawienie szybkości przetwarzania danego zdjęcia.
Na podstawie otrzymanego pliku zostały stworzone wykresy wizualizujące i umożliwiające
analizę i wyciągnięcie wniosków z przeprowadzonych badań.

%wrzucić zdjęcia pokazujące przykładowe wyniki detekcji linii dla różnych rodzajów obrazu - 3
%zdjęcia z przed i po



\section{Porównanie czasu działania dla implementacji szeregowej, wielowątkowej
oraz z wykorzystaniem biblioteki OpenCL}
\indent W celu przetestowania szybkości opracowanego algorytmu wykorzystującego
kartę graficzną, porównano go z dwoma innymi implementacjami algorytmu Viterbiego.
Dodatkowo kod programu został skompilowany używając trzech różnych kompilatorów C/C++
\begin{itemize}
    \item Microsoft Visual C++
    \item LLVM Clang
    \item MinGW
\end{itemize}
Dla każdego z nich w parametrach kompilacji została ustawiona optymalizacja szybkości
wykonywania programu : opcja \code{-O2}. Dzięki temu zaobserwowano diametralny wzrost
szybkości wykonywania algorytmów wykorzystujących wyłącznie CPU. W związku z tym 
zdecydowana się na porównanie wersji korzystającej z GPU, tylko z maksymalnie
zoptymalizowanymi pozostałymi algorytmami, w celu nałożenia surowszych wymagań
na wydajność rozpatrywanej implementacji.
\indent Algorytmy były porównywane na podstawie szybkości przetwarzania zdjęć
w zależności od ich rozmiaru oraz zakresu lokalnego sąsiedztwa $g\in \langle g_l, g_h \rangle$
(patrz rozdział \ref{viterbi_line}). 
%tu wrzucić wyniki z lapka
%tabela z wynikami

%wykresy


\section{Porównanie szybkości algorytmów dla różnych konfiguracji sprzętowych}
%porównać tu wyniki z pc'ta, dodać wykres porównujący średni
% czas wykonania algorytmu dla każdej z implementacji - nanieść/osobno wykres lapka i wykres pc'ta


\end{document}