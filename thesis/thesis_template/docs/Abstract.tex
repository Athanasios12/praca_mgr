\documentclass[document.tex]{subfiles}
\begin{document}
\clearpage
\begin{flushleft}
\textbf{\Huge{Streszczenie pracy}}
\vspace{1cm}
\\
W pracy przedstawiono teorię dotyczącą metod współbieżnego przetwarzania danych,
w ramach zastosowań przemysłowych i dziedzin systemów wbudowanych. Głównym założeniem pracy
była implementacja współbieżnego algorytmu Viterbiego do detekcji linii na obrazie cyfrowym.
Opisane zostały podstawy architektury procesorów, metody programowania równoległego
z ich wykorzystaniem. Przedstawiono zasadę działania algorytmu Viterbiego jako dekodera kodów splotowych
oraz jego adaptację do zastosowań przetwarzania obrazu. Dokonano zestawienia różnych wersji implementacji 
algorytmu oraz przeprowadzono badania porównujące ich szybkość dla różnych konfiguracji sprzętowych 
oraz parametrów wejściowych. W toku badań korzystano ze środowiska Visual Studio i kompilatora Visual C++, 
biblioteki OpenCL oraz Matplotlib dla języka Python
\\
\vspace{1cm}
\textbf{\Huge{Abstract}}
\vspace{1cm}
\\
Following thesis presents theory of concurrent data processing in industrial and embedded systems applications.
Main topic involves implementation of parallel Viterbi algorithm for line detection in digital image processing.
Thesis describes processors architecture and methods of concurrent programming. Additionally it presents Viterbi algorithm
principles as convolutional codes decoder and it's variation for image processing purposes. Furthermore, there is
a comparison of different versions of the algorithm, based on it's speed for multiple hardware configurations and input
parameters. During performed tests and there were used tools such as Visual Studio environment and Visual C++ compiler,
OpenCL and Matplotlib libraries.
\end{flushleft}
\clearpage
\end{document}