\documentclass[document.tex]{subfiles}
\begin{document}

\chapter{Algorytm Viterbiego}
\section{Opis działania i zastosowania}
\indent Alogorytm Viterbiego został stworzony i przeanalizowany przez A.J. Viterbiego w 1967r. Jego zadaniem było dekodowanie kodów splotowych. Później odkryto że posiada cechy programowanie dynamicznego i wykorzystuje maksymalne prawdopodobieństwo do określenia optymalnego zestawu tranzycji pomiędzy stanami. Algorytm został oryginalnie opracowany z myślą o zastosowaniach w telekomunikacji, ale znalazł zastosowanie w innych dziedzinach, m.in. w przetwarzanie obrazów, lokalizacji i rozpoznawaniu obiektów.
\\
\indent Główne zastosowanie algorytmu Viterbiego polega na
dekodowaniu informacji zakodowanych przy pomocy kodów splotowych. Sekwencja kodowana $m = m_1, m_2,...,m_n$, gdzie 
$m_i$ reprezentuje pojedynczy bit informacji, a indeks oznacza kolejność przesyłania. Enkoder splotowy przekształca informację wejściową w zakodowaną sekwencję $U = G(m)$.
Wykorzystuje do tego rejestr przesuwający, sumatory modulo oraz wielomiany generatora(\textit{generator polynomials}) określające związek sumatorów z rejestrem. 
\\
\indent Enkoder splotowy należy do klasy urządzeń zwanych automatami skończonymi(\textit{finite-state machines}), które zachowują informację o poprzednich sygnałach.
Stan enkodera w chwili %...
%dokonczyc o enkoderze

%indent dekodowanie - o drzewach trellis i omówic przykład + rysunek

%opisać rozpatrywany przypadek wykorzystania viterbiego
%do wyznaczania linii - arytkuły mazurek

%i chyba styka
\section{Implementacja w języku C++}
\indent Tworząc aplikację wykorzystującą algorytm Viterbiego do lokalizacji linii
na obrazie cyfrowym skorzystano z nowych funkcjonalności standardu C++11. 
Zdjęcia na, których szukano linii były wczytywane używając funkcji biblioteki
CImg. Reprezentowane przez obiekty \code{Cimg<T>}, dane pikseli zdjęcia były kopiowane do
dynamicznie zaalokowanych tablic jednowymiarowych obiektu \code{unique\_ptr}. Skorzystano z obiektu \code{unique\_ptr} w celu przechowywania wskaźnika do dynamicznie zaalokowanej pamięci, ze względu na funkcję automatycznego zwalniania zaalokowanych zasobów po wywołaniu destruktora \code{unique\_ptr}.  

%opis listinigu opencl_viterbi.cpp
\subsection{Wersja szeregowa}

%listing z viterbiLineDetect - nie pokazywac całego
\lstinputlisting[language={C++}, label={lst:viterbi_serial}, caption=Szeregowa implementacja algorytmu Viterbiego do wykrywania linii, linerange=224-297, firstnumber=224]{viterbi_source/Viterbi.cpp} 

%fragment inicjalizacja
\lstinputlisting[language={C++}, label={lst:viterbi_serial}, caption=Szeregowa implementacja algorytmu Viterbiego do wykrywania linii, linerange=224-240, firstnumber=224]{viterbi_source/Viterbi.cpp} 


%opisać wszystkie fragmenty z odniesieniem do schematu ogólnego algorytmu - shcemat z atykułów i wzory
\subsection{Wersja równoległa - C++11}
To jest podrozdział 2 rozdziału 2
\subsection{Wersja równoległa - OpenCL}
To jest podrozdział 3 rozdziału 2
\end{document}