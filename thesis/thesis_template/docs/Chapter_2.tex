\documentclass[document.tex]{subfiles}
\begin{document}

\chapter{Algorytm Viterbiego}
\section{Opis działania i zastosowania}
To jest rozdział 1
\section{Implementacja w języku C++}
\indent Tworząc aplikację wykorzystującą algorytm Viterbiego do lokalizacji linii
na obrazie cyfrowym skorzystano z nowych funkcjonalności standardu C++11. 
Zdjęcia na, których szukano linii były wczytywane używając funkcji biblioteki
CImg. Reprezentowane przez obiekty \code{Cimg<T>}, dane pikseli zdjęcia były kopiowane do
dynamicznie zaalokowanych tablic jednowymiarowych obiektu \code{unique\_ptr}. Skorzystano z obiektu \code{unique\_ptr} w celu przechowywania wskaźnika do dynamicznie zaalokowanej pamięci, ze względu na funkcję automatycznego zwalniania zaalokowanych zasobów po wywołaniu destruktora \code{unique\_ptr}.  

%opis listinigu opencl_viterbi.cpp
\subsection{Wersja szeregowa}

%listing z viterbiLineDetect - nie pokazywac całego
\lstinputlisting[language={C++}, label={lst:viterbi_serial}, caption=Szeregowa implementacja algorytmu Viterbiego do wykrywania linii, linerange=224-297, firstnumber=224]{viterbi_source/Viterbi.cpp} 

%fragment inicjalizacja
\lstinputlisting[language={C++}, label={lst:viterbi_serial}, caption=Szeregowa implementacja algorytmu Viterbiego do wykrywania linii, linerange=224-240, firstnumber=224]{viterbi_source/Viterbi.cpp} 


%opisać wszystkie fragmenty z odniesieniem do schematu ogólnego algorytmu - shcemat z atykułów i wzory
\subsection{Wersja równoległa - C++11}
To jest podrozdział 2 rozdziału 2
\subsection{Wersja równoległa - OpenCL}
To jest podrozdział 3 rozdziału 2
\end{document}