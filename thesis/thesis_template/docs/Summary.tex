\documentclass[document.tex]{subfiles}
\begin{document}
\chapter{Wnioski końcowe}
\indent Opracowane oraz analiza zagadnień teoretyczne dotyczące współbieżności oraz architektury procesorów
umożliwiła lepsze rozeznanie oraz wyciągnięcie wniosków z przeprowadzonych badań. Zestawione metody
implementacji algorytmu Viterbiego pokazały ich możliwości, wady oraz zalety w zastosowaniu 
do obliczania w czasie rzeczywistym położenia linii na obrazie cyfrowym. Głównym celem pracy 
było przeprowadzenie badań mających na celu pomoc w wyborze odpowiedniej implementacji
algorytmu Viterbiego w sterowaniu robotem mobilnym ze sprzężeniem wizyjnym.
Na podstawie otrzymanych wyników można zauważyć, żę dla wszystkich testowanych zdjęć zakres sąsiedztwa
$g\in \langle -4, 4 \rangle$ umożliwia dokładne wykrycie linii nawet na bardzo zaszumionym obrazie 
wejściowym.
\\
\indent Dodatkowo jeśli chodzi o szybkość przetwarzania zdjęć nie jest opłacalna analiza 
całego obrazu dla każdego przechwytywanego obrazu w celu sterowania w czasie rzeczywistym.
Bardziej opłacalna jest analiza tylko fragmentu zdjęcia potrzebnego do określenia nowego
położenia robota, gdzie ilość rozpatrywanych kolumn, bądź wierszy odpowiada szybkości 
kamery oraz założonej prędkości ruchu robota. W celu możliwości zastosowania
opracowanych implementacji potrzebna jest ich modyfikacja tak aby dodatkowo zależały
one od parametrów kamery oraz robota mobilnego.
\\
\indent Analizując otrzymane wyniki w rozdziale \ref{chapter_3}, można wywnioskować
że w celu osiągnięcia jak najszybszej dokładnego położenia linii prowadzącej ruch robota 
mobilnego, niezbędne jest zastosowanie osobnego urządzenia wyłącznie przeznaczonego
do przeprowadzania obliczeń sterowania. Aby ograniczyć pobór prądu nie powinno być ono
podłączone do źródła zasilania robota, tylko powinno się komunikować z nim bezprzewodowo.
Procesor robota mobilnego odpowiedzialny byłby tylko za akwizycję obrazu z kamery oraz
sterowanie silnikami.
\\
\indent W celu uzyskania lepszych osiągów warte jest przetestowanie opracowanych
algorytmów dla GPU, używając nowocześniejszych kart graficznych, dedykowanych do obliczeń równoległych,
takich jak NVIDIA Titan\cite{titan_spec}. Dodatkowo chcąc zwiększyć szybkość implementacji hybrydowej i wielowątkowej
warto jest zastosować CPU z rodziny procesorów Intel Xeon(do 14 rdzeni i 48 wątków systemowych\cite{xeon_spec}).
\end{document}