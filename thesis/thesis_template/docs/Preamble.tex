\documentclass[document.tex]{subfiles}
\long\def\/*#1*/{}
\begin{document}
\chapter{Wstęp}

\section{Przetwarzanie obrazu i jego rola w automatyce przemysłowej}
	\indent W zagadnieniach technik pomiarowych oraz analizy otoczenia coraz częściej stosowane
	są rozwiązania wykorzystujące systemy wizyjne. Do najpopularniejszych zastosowań przemysłowych wizji maszynowej należą \cite{Machine_Vision_Intro}:
	\begin{itemize}
		\item inspekcja elementów na linii technologicznej
		\item określanie właściwej orientacji i położenia elementów
		\item identyfikacja produktów
		\item pomiary metrologiczne		
	\end{itemize}
	\indent W automatyce przemysłowej gdzie do zagadnień inspekcji
	wcześniej niezbędna była ocena wizualna człowieka, obecnie powszechnie stosuje się systemy wizyjne, 
	w których skład wchodzą kamery przemysłowe, czujniki wyzwalające(np. na bazie pozycji) oraz urządzenie odpowiadającego za proces decyzyjny.
	Występują również rozwiązania w postaci systemów wbudowanych, gdzie inteligentna kamera oprócz
	akwizycji obrazu zajmuje się jego przetwarzaniem i analizą, wykorzystując własny procesor.\cite{Machine_Vision_Intro}\cite{Davies_Machine_Vision}
	
	% obrazek : może scieżkę dodac wczesniej w mystyle albo w dokument.tex, wtedy odwoływac się tylko do nazwy

%	\begin{figure}[h]
%	\includegraphics[width=0.5\textwidth, inner]{imgs/inspection_img}
%	\caption{Przykład zautomatyzowanej linii technologicznej wykorzystującej system wizyjny\cite{vision_systems_article}}
%	\label{fig:inspekcja}
%	\end{figure}

	\indent Sprawdzanie orientacji i położenia elementów w przemyśle jest wykorzystywane
	między innymi w technologii montażu, gdzie informacje z urządzeń wizyjnych są wykorzystywane
	przez manipulatory przemysłowe do zautomatyzowanego montażu, sortowania oraz paletyzacji wyrobów.\cite{Machine_Vision_Intro}

	%obrazek - sprawdzanie orientacji
%	\begin{figure}[h]
%	\includegraphics[width=0.5\textwidth, inner]{imgs/pos_oritent_img}
%	\caption{Przykład obrazów używanych w testowaniu pozycji i orientacji elementów\cite{Machine_Vision_Intro}}
%	\label{fig:pozycja_orientacja}
%	\end{figure}	

	\indent Identyfikowanie produktów na bazie obrazu cyfrowego jest wykorzystywane przy sortowaniu
	oraz monitorowaniu przepływu elementów i lokalizacji wąskich gardeł.
	Przykładowe metody indentyfikacji to stosowanie kodów kreskowych i kodów DataMatrix.\cite{Machine_Vision_Intro}

	%identyfikacja
%	\begin{figure}[h]
%	\includegraphics[width=0.5\textwidth, inner]{imgs/identyfication_img}
%	\caption{Przykład wizyjnej identyfikacji\cite{Machine_Vision_Intro}}
%	\label{fig:identyfikacja}
%	\end{figure}

\section{Istotność szybkości obliczeń w problemach wizji maszynowej}
%wstęp
\indent Większość praktycznych zastosowań przetwarzania obrazu jako dodatkowej 
informacji w sterowaniu jednym bądź grupą urządzeń, wymaga akwizycji oraz wykonywania
obliczeń w czasie rzeczywistym. Oznacza to, że wybrany algorytm wykorzystywany do 
analizy obrazu cyfrowego, wraz z resztą niezbędnego kodu, musi posiadać czas wykonania 
spełniający narzucone przez sterowany system.
%automatyka przemysłowa
\\ \indent Dla zastosowań przemysłowych. gdzie monitorowane obiekty poruszają się z dużą prędkością, szybkość podjęcia decyzji przez system wizyjny
może być wąskim gardłem dla danej gałęzi linii produkcyjnej. Czas na wykonanie decyzji
(np. o usunięciu wadliwego produktu z przenośnika taśmowego), składa się na czas akwizycji obrazu,
obliczenia sterowania. Standardowe kamery przemysłowe potrafią zrobić nawet powyżej 100 zdjęć na sekundę, a nawet więcej stosując mniejsze rozdzielczości obrazu. Czas przesyłu danych dla
standardu popularnego standardu GigE wynosi maksymalnie 125 MB/s \cite{NI_camera_buses}. Na podstawie tego można stwierdzić, że główny problem będzie stanowił czas obliczenia sterowania i od niego będzie zależeć szybkość działania systemu wizyjnego \cite{Cognex_catalog\cite{Basler_Catalog}.
%robotyka mobilna
\indent Inną dziedziną gdzie stosowane jest przetwarzanie obrazu w czasie rzeczywistym
jest robotyka mobilna, gdzie system wizyjny może odpowiadać za:
\begin{itemize}
	\item Sprzężenie niezbędne do obliczenia zmiany położenia i prędkości robota.
	\item Lokalizację przeszkód oraz innych robotów(Swarm Robotics}.
	\item Analizę oraz monitorowanie otoczenia.
\end{itemize}
%dopisać o krótko o budowie robota mobilnego oraz czemu czas rzeczywisty

%opisać możliwe rozwiązania dla urządzeń decyzyjnych systemu wizyjnego
%mikrokontrolers, embedded processors, conventional multiore cpus,  fpgas, gpus
podrozdział 2 \cite{Mazurek_Robot_Viterbi}
\section{Cel, zakres i zastosowania pracy}
podrozdział 3
\end{document}